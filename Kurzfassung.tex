\chapter{Kurzfassung}
Die Software EMS soll in jeder Situation ein professionelles Eventmangement ermöglichen.
Die Kernaufgaben liegen darin, dass Benutzer und Tickets zusammen mit Events verwaltet werden können.
Sogenannte Influencer und Promoter, welche hauptsächlich Personen mit einer großen Reichweite auf Social Media Kanälen sind, ist die neueste Art von Marketing und verspricht für ein Event schnell, viel Aufmerksamkeit zu bekommen.
Diese App soll eine Plattform bieten, diese Influencer und Promoter leicht und überschaulich zu verwalten und das Marketing eines Event dadurch besser zu koordinieren.
Weiters soll durch die Applikation der Eintrittsticketverkauf und der Prozess dahinter, vor allem organisatorisch verbessert und übersichtlicher gemacht werden. 
Die Software unterscheidet einen Benutzer anhand von zwei Rollen, eine Rolle ist die des Promoters, die andere ist die eines Administrators.

Die Anwendung gliedert sich in drei verschiedene Hauptseiten auf, wovon zwei einem normalen Promoter zugänglich sind.
Jeder Promoter hat ein Profil mit seinem Namen, einem Profilbild und einer Beschreibung. Dieses Profil können die Promoter selbst anpassen. Der Name ist jedoch nicht von einem normalen Promoter änderbar.
Die Hauptseite umfasst die Übersicht aller Events. Hier werden alle dem Promoter zugeteilten Events, für welche er Karten verkaufen soll, angezeigt.
Er kann auf eines der Events klicken und gelangt dadurch auf eine detailliertere Seite der gewählten Veranstaltung.
Hier kann er seinen Kartenstand aktualisieren. Für jede verkaufte Karte oder jedes verkaufte Package (Sammlung von Karte) bekommt ein Promoter sogennante Goodie-Punkte.
Diese Punkte kann er dann auf der Event-Details Seite für bestimmte Goodies eintauschen, wie beispielweise \textbf{Ein gratis Getränk beim Event für 10 Punkte}.

Die letzte Seite ist das Admin-Terminal. Wie der Name schon sagt steht diese Seite nur einem Administrator zur Verfügung.
Hier kann dieser einen Benutzer erstellen, ihn deaktivieren und aktivieren und einem Benutzer einem Event zuweißen.
Weiters werden hier auch die Events erstellt indem man Kartentypen, Packages und Belohnungen für die Goodie-Punkte zum einlösen, festlegt.

\textbf{Hinweis:} Sämtliche angegebenen Quellen und Links wurden mit Stand 20.04.2021, 16:00 auf Ihre Richtigkeit und Integrität überprüft.
Sie entsprechen dem momentanen Stand des Wissens in ihren jeweiligen Gebieten zu gegebenem Datum.
Änderungen der Quellen nach angegebenem Zeitpunkt wurden nicht in dieses Dokument übernommen.
Falls solche stattfinden ist das Dokument als veraltet zu betrachten und muss aktualisiert werden.
Die Daten bei den Quellen wurden mithilfe eines selbst erstellten JavaScripts auf ihre aktualität überprüft.