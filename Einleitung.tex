\chapter{Einleitung}


\section{Ausgangslage}
Aktuell gestaltet sich die Verwaltung der Switches im Werk der Essity Austria GmbH in Ortmann kompliziert, da es in einem Netzwerk dieser Größe sehr viele Fehlerquellen geben kann. Die Fehlersuche ist sehr mühsam und nimmt sehr viel Zeit in Anspruch. Oft ist die Suche nach dem Gerät, welches den Fehler verursacht zeitaufwändiger und mühsamer, als die eigentliche Fehlerbehebung. Im Moment muss immer ein beziehungsweise mehrere Mitarbeiter zu dem verdächtigten Switch gehen und vor Ort versuchen den/die Fehler zu finden und zu beheben.

\section{Ziele}
Ziel des Projekts ist es, die Fehlersuche im Unternehmen zu optimieren und somit den Zeitaufwand erheblich zu verkürzen. Um dieses Ziel realisieren zu können, wird ein System entwickelt, mit dem man alle Switches des gesamten Werkes an einem Ort zentral überwachen und verwalten kann. Die Kommunikation zu den Switches wird über Telnet oder SSH erfolgen, je nach dem, mit welchem Switch man kommunizieren will. Weiters soll die Kommunikation mittels SSL verschlüsselt sein. Des Weiteren soll Single-Sign-On als Authentifizierungsmethode verwendet werden. Die Zugriffsrechte auf das System werden mittels Acitve-Directory-Gruppen geregelt. 
\newline

Der Zeitaufwand für die Wartung bzw. Fehlersuche innerhalb der Netzwerkstruktur im Unternehmen wird um ein Vielfaches verkürzt, da man nicht mehr direkt vor Ort sein muss. Mit ein paar einfachen Mausklicks hat man alle Switches im Blick und kann die relevanten Parameter überprüfen. Zurzeit hängt der zeitliche Aufwand der Fehlersuche oft mit Glück zusammen und kann entweder 20 Minuten oder mehrere Stunden dauern. Durch das System soll der Aufwand so gering wie möglich gehalten werden und führt zu einer effizienteren Nutzung der verfügbaren Arbeitszeit.

\newpage
\section{Team}

\subsection{Tim Alderkot}
\subsubsection{Themenstellung}
Untersuchung und Vergleich unterschiedlicher Testverfahren für Webapplikationen unter besonderer Berücksichtigung des Performance- und Lastverhalten
\subsubsection{Rolle}
In diesem Projekt übernahm er die Rolle als Applikations-Tester. Er hatte die Verantwortung für die korrekte Funktionalität des Systems. Außerdem überprüfte er das Lastverhalten der Applikation, als auch die Performance.

\subsection{Marco Postl}
\subsubsection{Themenstellung}
Grundlagen und Einsatzgebiete sowie Vergleich aktueller Webframeworks mit dem Schwerpunkt auf grafische Oberflächen
\subsubsection{Rolle}
In diesem Projekt übernahm er die Rolle als Frontend-Entwickler. Er war für die Entwicklung des Frontends verantwortlich, sowie für den Support bei Problemen in der Entwicklung des Backends.

\subsection{Manuel Scheibenreif}
\subsubsection{Themenstellung}
Grundlagen der Netzwerksicherheit mit Fokus auf SSH sowie Vergleich mit anderen Protokollen
\subsubsection{Rolle}
In diesem Projekt übernahm er die Rolle als Teamleiter. Er koordinierte das Projektteam und hielt Kontakt mit dem Auftraggeber. Des Weiteren war er für die Entwicklung des Backends verantwortlich, sowie einiger Teile des Frontends. 

