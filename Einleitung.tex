\chapter{Einleitung}
\section{Ausgangslage}
Bei Veranstaltungen mit einer Alterszielgruppe von 16 bis 21 jährigen wird heutzutage oft auf Marketing mit Influencer und Promoter gesetzt. 
Sogenannte Influencer und Promoter sind hauptsächlich auf Instagram, Facebook und anderen Social-Media Netzwerken tätig und bewerben dort das jeweilige Event. 
Nachdem eine Nachfrage nach Eintrittskarten besteht, ist es die Aufgabe der Promoter den Eintrittskartenvertrieb zu übernehmen. 

Bei diesen Events müssen 100-150 Promoter Karten erhalten und verkaufen und zu einem späteren Zeitpunkt, das eingenommene Geld abgegeben. Dadurch entsteht eine großer 
administrativer Aufwand. Aktuell wird alles mit Hand geführten Protokollen und Listen dokumentiert.
Dies sorgt bei dieser Anzahl an Daten schnell für Fehler, wobei es auch lange dauert, um den Fehler zu finden.
Für die Auswertung der gesammelten Daten wird eine hochkomplexe Excel Liste benutzt. Leider ist dieser Prozess oft ungenau und wichtige 
Daten gehen verloren, oder der Verlauf ist später nicht mehr nachvollziehbar. 

\section{Ziele}
Durch eine iOS und Android App soll in erster Linie den Promoter-Managern und der Geschäftsleitung viel organisatorischen Aufwand abnehmen. 
Das System hat Administrator Accounts,  welche der Gruppe Admin angehören und dann die Benutzer der Gruppe Promoter verwaltet.
Die Promoter verkaufen Karten und müssen für das Event, welches sie zugeteilt bekommen, Werbung machen.

Wenn ein Promoter eine Karte verkauft, muss er seinen Kartenbestand, der in der App gespeichert ist, aktualisieren und bekommt dadurch ebenfalls Punkte gutgeschrieben.
Diesen Kartenstand muss er mindestens einmal pro Tag aktualisieren, da er sonst Punkte verliert, auch wenn er keine Karten verkauft hat.
Somit hat die Geschäftsleitung immer einen aktuellen Stand der Kartenverkäufe und der Punkte der Promoter.

Ab einer bestimmten Punkteanzahl können die Promoter diese für bestimmte Goodies wie Freikarten oder Getränkegutscheine bekommen.
Weiteres kann jeder Promoter mit einem Admin, bei eventuell auftretenden Fragen, privat in der App schreiben.
Die Software wird insgesamt also einige Schritte erleichtern, mehr Sicherheit, vor allem auch im Bereich Datenschutz bieten und einige Statistiken und Datenerhebungen für zukünftige Events liefern,
welche bis jetzt in so einem Umfang noch nicht möglich waren.

\newpage
\section{Team}
\subsection{Maurice Putz}
\subsubsection{Themenstellung}
Chatsystem in EMS, Cloud Computing und Beispiel eines Backend Deployment in AWS mit Schwerpunkt auf Grundlagen im Datenschutz und DSGVO Konformität
\subsubsection{Rolle}
In diesem Projekt übernahm er die Rolle eines Fullstack Entwickler. Er hatte die Verantwortung für das Chatsystem,
sowie zahlreiche Funktionen im Front- sowie Backend. Außerdem war er für das Deployment des Backend der App am Ende zuständig.

\subsection{Benjamin Strahlhofer}
\subsubsection{Themenstellung}
Anmeldesystem in EMS, Abwicklung von Sicherheitsrisiken und Vergleich mit Biometrie, mit dem Schwerpunkt auf Sicherheit eines Anmeldesystems
\subsubsection{Rolle}
In diesem Projekt Übernahm er die Rolle eines Fullstack Entwickler und Projektleiter. Er hatte die Verantwortung für das Anmeldesystem und 
sowie zahlreiche Funktionen im Front- und Backend sowie die Koordination des Teams und der User-Stories in den Sprints.

\subsection{Thomas Bauer}
\subsubsection{Themenstellung}
Allgemeines zu Data Analytics und Grundlagen von Rest API's, Vergleich von SQL und einer NoSQL Datenbank mit Schwerpunkt auf Datenbankentwurf
\subsubsection{Rolle}
In diesem Projekt Übernahm er die Rolle eines Fullstack Entwickler. Er hatte die Verantwortung über die Datenbank, 
sowie zahlreiche Funktionen im Front- und Backend. 

\subsection{Alexander Reiter}
\subsubsection{Themenstellung}
Ticketsysteme, Untersuchung von Belohnungssystemen für Wettbewerbssituationen und Vorgehensmodelle mit Schwerpunkt auf einem Vergleich von SCRUM und RUP
\subsubsection{Rolle}
In diesem Projekt übernahm er die Rolle als Applikationstester und Fullstack Entwickler. Er ist außerdem für das Ticketsystem und dessen Funktionalität
verantwortlich, sowie auch bei zahlreichen anderen Funktionen im Front- und Backend.
\newpage