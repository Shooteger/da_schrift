\chapter{Datenschutzgrundverordnung (DSGVO)}
\putz

%Quellen:
%Reis Folien
%https://de.wikipedia.org/wiki/Datenschutz-Grundverordnung
%https://wirtschaftslexikon.gabler.de/definition/datenschutz-grundverordnung-99476
\section{Aufbau und Inhalt}
\subsection{DSGVO Grundsätze}
Die Datenschutzgrundverordnung, oder im englischen Sprachraum \textbf{General Data Protection Regulation (GDPR)} genannt, soll die Rechte von natürlichen, realen Personen im Bezug auf Verarbeitung deren personenbezogenen Daten, genau definieren und sicherstellen. Die DSGVO wurde am 25. Mai 2016 beschlossen und trat in Kraft. Sie musste ab diesem Zeitpunkt unter Einhaltung einer zwei jährigen Umsetzungsfrist bis 25. Mai 2018 eingehalten werden. Primär soll jedem EU-Bürger das Recht auf informelle Selbstbestimmung garantiert werden und jeder Mensch in der EU soll vor nicht sachgerechter Nutzung der eigenen Daten rechtlich geschützt werden.
Die neue Datenschutz Richtlinie der Europäischen Union ersetzt die bis dahin geltenden \textbf{Datenschutzrichtlinie 95/94}. Das Gesetz wurde als EU Verordnung erlassen und somit in allen Mitgliedsstaaten vollständig übernommen werden. Die Verordnung enthält jedoch 69 sogenannte Öffnungsklauseln, welche den jeweilige Staate der Europäischen Union eine nationale Selbstbestimmung in diese Punkten erlaubt. Zum Beispiel das alter, ab welchem das Gesetz für eine Person gilt. Hier gilt ein Spielraum, dass das Alter, ab wann die DSGVO auf ein Individuum zutrifft, national selbstbestimmt werden kann, jedoch nicht über 16 und nicht unter 13 liegen darf.

Artikel 2, der sachliche Anwendungsbereich der DSGVO beinhaltet:
\begin{itemize}
	\item ganz oder teilweise automatisierte Verarbeitung von personenbezogener Daten
	\item manuelle Verarbeitung personenbezogener Daten, welche in einem Dateisystem gespeichert werden
\end{itemize}

Für die Verordnung gibt es gewisse Ausnahmen (Artikel 1, Abs. 2):
\begin{itemize}
	\item im Rahmen von Tätigkeiten, welche nicht in den Anwendungsbereich des Unionsrechts fallen
	\item Tätigkeiten im Rahmen der gemeinsamen Außen- und Sicherheitspolitik (Titel V, Kapitel 2 EUV)
	\item durch natürliche Personen zur Ausübung ausschließlich persönlicher oder familiärer Tätigkeiten
	\item durch die zuständigen Behörden zum Zwecke der Verhütung, Ermittlung, Aufdeckung oder Verfolgung von Straftaten oder der Strafvollstreckung, einschließlich des Schutzes vor und der Abwehr von Gefahren für die öffentliche Sicherheit
\end{itemize}

\subsection{Kapitel und Aufbau der DSGVO}
Die Datenschuttgrundverordnung ist in 11 Kapitel unterteilt. Diese Kapitel sind wiederum in einzelne Artikel unterteile, 99 an der Zahl. Eine Übersicht:

\begin{itemize}
	\item \textbf{Kapitel I (Artikel 1 bis 4)}: Allgemeine Bestimmungen
	\item \textbf{Kapitel II (Artikel 5 bis 11)}: Grundsätze und Rechtmäßigkeit
	\item \textbf{Kapitel III (Artikel 12 bis 23)}: Grundsätze und Rechtmäßigkeit
	\item \textbf{Kapitel IV (Artikel 24 bis 43)}: Verantwortlicher und Auftragsverarbeiter
	\item \textbf{Kapitel V (Artikel 44 bis 50)}: Übermittlungen personenbezogener Daten an Drittländer oder an internationale Organisationen
	\item \textbf{Kapitel VI (Artikel 51 bis 59)}: Unabhängige Aufsichtsbehörden
	\item \textbf{Kapitel VII (Artikel 60 bis 76)}: Zusammenarbeit und Kohärenz, Europäischer Datenschutzausschuss
	\item \textbf{Kapitel VIII (Artikel 77 bis 84)}: Rechtsbehelfe, Haftung und Sanktionen
	\item \textbf{Kapitel IX (Artikel 85 bis 91)}: Vorschriften für besondere Verarbeitungssituationen
	\item \textbf{Kapitel X (Artikel 92 bis 93)}: Delegierte Rechtsakte und Durchführungsrechtsakte
	\item \textbf{Kapitel XI (Artikel 94 bis 99)}: Schlussbestimmungen
\end{itemize}

\subsection{Definitionen}
Bestimmte Fachbegriffe wie "`natürliche Peron"' oder was sind personenbezogene Daten, sind in der EU Verordnung definiert und beschrieben worden.

\paragraph{}

%Quellen:
%https://www.privacypolicies.com/blog/privacy-by-design/
%https://gdpr-info.eu/issues/privacy-by-design/
%https://keyed.de/blog/software-dsgvo/#DSGVO%20Checkliste%20f%C3%BCr%20Software
\section{Capgemini studie 2019 und 2020}

\newpage
