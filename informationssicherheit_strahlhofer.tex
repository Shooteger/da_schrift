\chapter{Grundlagen der Informationssicherheit}
\strahlhofer

%Reis seine erster Foliensatz
\section{Allgemeines zu Informationssicherheit}
\subsection{Definition}
Bei der Informationssicherheit geht es grundsätzlich darum Informationen zu schützen. Informationen spielen in der heutigen Welt eine wichtige Rolle. Die zwei Arten Informationen zu verarbeiten sind entweder in einer digitalen oder analogen Form. 
\\
Weiters wird überprüft, dass eine fremde, nicht befugte Person oder ein nicht befugtes Unternehmen, auf keinen Fall den Zugriff auf die unternehmensinternen Systeme und Benutzerdaten erhält. Ein Unternehmen muss dafür Maßnahmen setzen, um die Nutzer ihres Services zu schützen. Diese Maßnahmen sollen verschiedenste Bedrohungen vermeiden und zur Datensicherheit, IT-Sicherheit und Datensicherung beitragen.
\\

\section{BSI}

%https://de.wikipedia.org/wiki/ISO/IEC-27000-Reihe
%https://www.onlinesicherheit.gv.at/Themen/Experteninformation/Normen-und-Standards/ISO-IEC-27000.html
\section{ISO/IEC-27000-Reihe}
Die ISO/IEC 27000-Reihe ist eine Ansammlung von Standards über das Informationssicherheitsmanagement (ISMS) und gibt Ratschläge über gewisse Best Practices in diesem Bereich. Sie wurden von der International Organization for Standardization (ISO) und der International Electronic Commission (IEC) publiziert. 
Diese Zertifizierungs-Reihe ist so absichtlich so breit beschrieben, da es für alle möglichen Firmen in jeglicher Größe und Form anwendbar ist.

\subsection{Geschichte}



%https://www.bsi.bund.de/SharedDocs/Downloads/DE/BSI/Grundschutz/Kompendium/Elementare_Gefaehrdungen.pdf?__blob=publicationFile&v=4
%Reis Foliensatz 1
\section{Bedrohungen für Informationssicherheit}
In der Informationssicherheit gibt es viel verschiedene Arten von Bedrohungen die sich laufend verändern. Im BSI sind 47 unterschiedliche Arten von Bedrohungen in der Informationssicherheit aufgelistet.

%bsi und folien von reis
\begin{itemize}
	\item \textbf{Gefährdungen durch natürliche Ereignisse} darunter fallen Naturkatastrophen, Feuer, Wasser, Verschmutzungen durch Staub, Korrosion und ungünstige klimatische Bedingungen die zur Zerstörung der IT-Infrastruktur führen.
	\item \textbf{Ausfälle oder Störungen} diese können von der Stromversorgung, Kommunikationsnetzen, Versorgungsnetzen oder Dienstleistern kommen
	\item \textbf{Schadprogramme} beispielsweise Viren, Würmer, Trojaner, Adware, Spyware, Ransomware
	\item \textbf{Verhinderungen von Diensten (Denial of Service)} auch DoS-Angriffe genannt, versuchen Störungen von Geschäftsprozessen oder IT-Ausfälle herbeizuführen.
	\item \textbf{Social Engineering}, hier gibt sich eine Person als Administrator der eigenen Firma aus. Dieser Angreifer versucht beispielsweise per Telefonat einen Mitarbeiter durch Leichtgläubigkeit dazu  sein Anmeldeinformationen auszugeben.
	\item \textbf{Identitätsdiebstahl} beispielsweise Phishing ist eine Technik die Angreifer verwenden um Kundendaten zu erlangen. Es wird unter anderem eine Anmeldeseite erstellt die identisch zur eigenen ist, um Kunden im Glauben zu lassen, dass sie nichts falsch machen.
	\textbf{Ressourcenmangel} kann auftreten wenn alte Hard- oder Software verwendet wird. 
\end{itemize}

%folien
\section{Gründe für das Informationssicherheitsmanagement}
\paragraph{Gesetzliche Notwendigkeit}
Es liegt eine gesetzliche Notwendigkeit im Sinne des Treffens der geeigneten technischen und organisatorischen Maßnahmen vor, beispielsweise die EU-DSGVO, das Telekommunikationsgesetz, das Urheberrechtsgesetz.

\paragraph{Wettbewerbsvorteil}
Wenn eine Firma einen hohen Stand an Informationssicherheit besitzt, kann sie Konkurrenzfähig bleiben. Zusätzlich kann man mit Partnerunternehmen ins Geschäft kommen, die einen hohen Wert auf Informationssicherheit setzen. Durch Zertifikate kann nachgewiesen werden, dass der Betrieb über ein hohes Sicherheitsniveau verfügt.

\paragraph{Risiken identifizieren}
Ein Vorteil für ein Unternehmen ist das Risiken die mit den Bedrohungen der Informationssicherheit zu tun habe, eher erkannt werden und infolgedessen auch Schäden umgangen werden können.

%https://wotan-monitoring.com/de/news/news-detail/die-schutzziele-der-informationssicherheit-verfuegbarkeit-integritaet-und-vertraulichkeit.html#:~:text=Informationssicherheit%20und%20ISO%2FIEC%2027001&text=Und%20im%20Hauptdokument%2C%20der%20ISO,Vertraulichkeit%E2%80%9C%20und%20deren%20Aufrechterhaltung%20definiert.
%Folien reis
%https://www.bsi.bund.de/SharedDocs/Downloads/DE/BSI/Grundschutz/Kompendium/IT_Grundschutz_Kompendium_Edition2020.pdf%3F__blob%3DpublicationFile%26v%3D6
\section{ISO/IEC 27001}
Die ISO/IEC 27001 definiert sogenannte Schutzziele oder Anforderungen an ein Informationsmanagementsicherheitssystem (ISMS). In dem Dokument wird genau auf die Einhaltung der Bereiche Vertraulichkeit(confidentiality), Integrität(integrity) und Verfügbarkeit(availability) eingegangen. Laut ISO/IEC 27000, gibt es weitere Faktoren die auch ein Teil der Informationssicherheit sein können, das sind Authentizität(authenticity), Zurechenbarkeit(accountability), Nicht-Abstreitbarkeit(non-repudiation) und Verlässlichkeit(reliability).

\subsection{Schutzziele der ISO/IEC 27001}
\paragraph{Vertaulichkeit}
Die Vertraulichkeit beschreibt die Eigenschaft, dass eine Information nicht zugänglich gemacht wird für eine nicht autorisierten Person oder Unternehmen. Somit ist der Schutz gegen Verrat oder Spionage gewährleistet. Darunter wird verstanden, dass Kennzahlen, wie Umsatzzahlen nicht an die Öffentlichkeit gelangen. 


\paragraph{Integrität}
Die Integrität laut dem BSI beschäftigt sich mit der Korrektheit von Informationen. In der Informationssicherheit wird damit assoziiert, dass eine unbefugte Person kein Daten manipulieren kann und darf. Darunter wird verstanden, das Veränderungen wie Datensätze einfügen, verändern oder löschen nicht möglich ist.
Im Rahmen von Integrationstest kann überprüft werden ob Daten von befugten Individuen manipuliert werden, beispielsweise kann dieser Aspekt mittels Hashfunktion überprüft werden.


\paragraph{Verfügbarkeit}
Unter Verfügbarkeit wird in diesem Schutzziel verstanden, dass ein IT-System oder eine Anwendung, für den Benutzer, dass kann ein Mitarbeiter oder ein Kunde sein, jederzeit zu Verfügung steht. Fälle wie, beispielsweise der Ausfall oder nicht Verfügbarkeit durch eine DDoS-Attacke  (Distributed Denial of Service) oder eine DoS-Attacke(Denial of Service) eines Bank-Systems. Wenn das System nicht oder teilweise funktioniert, können möglicherweise keine Überweisungen getätigt werden und folge dessen sind die Angestellten und Konsumenten eingeschränkt. 

%https://en.wikipedia.org/wiki/Man-in-the-middle_attack
\paragraph{Authentizität}
Bei diesem Schutzziel geht es darum, das ein System gewährleistet, dass ein Kommunikationspartner wirklich die Person ist, die er vorgibt zu sein. Bei der Überprüfung der Authentizität muss ein Benutzer sich vorerst authentisieren, dass kann durch viel verschiedene Methoden erfolgen, wie die Eingabe von Benutzername und Passwort oder die Sendung des Reisepasses oder eine Anmeldung via Fingerabdruck oder Gesichtserkennung. Danach kann durch ein Kontrollmodul die Identitätskontrolle an der Person durchgeführt werden und dadurch authentifiziert werden. Am Ende dieses Vorgangs ist eine Person im System angemeldet und es werden im Rahmen der Autorisierung bestimmte Rollen oder Rechte zugewiesen.
Dieses Schutzziel kann Man-in-the-Middle-Angriffe verhindern. Bei dieser Cyberattacke setzt sich eine Person zwischen die Kommunikation zweier verschiedener Parteien und kann damit Transaktionen manipulieren. 

%https://de.wikipedia.org/wiki/Informationssicherheit#Zugangskontrolle
\paragraph{Zugriffssteurung}
Als nächstes Schutzziel gibt es die Zugriffssteuerung. Diese wird gewährleistet wenn der Zugang zu Informationen oder Ressourcen nur autorisiert statt findet, somit wird ein Berechtigungskonzept umgesetzt. Das kann durch verwenden von "individuellen Benutzernamen oder komplexen Passwörtern" erfolgen oder durch eine Authentifizierung durch mehrere Faktoren. In der Praxis werden meist sogenannte Zwei-Faktor Authentifizierungsmethoden verwendet, dass kann beispielsweise durch den Versand eines Codes via Mail erfolgen.

%Reis Folie
\paragraph{Verlässlichkeit}
Bei der Verlässlichkeit geht es darum, dass ein System immer konsistent und nach den festgelegten Spezifikationen läuft. Darüber hinaus soll es zu jederzeit folgerichtige Ergebnisse liefern. Beispielsweise wenn ein System als Hauptfunktionalität das Passwörter verschlüsseln hat, darf es nicht dazu kommen, dass eine inkonsistente Verschlüsselung bei jedem zehnten Benutzer verwendet wird.

%https://en.wikipedia.org/wiki/Information_security#Non-repudiation
\paragraph{Nichtabstreitbarkeit/Verbindlichkeiten}
Dieses Schutzziel gibt an, dass eine Person die ihren Verpflichtungen und Aufgaben aus einem im Vorhinein festgelegten Vertrag nach geht. Des weiteren gilt das wenn ein Beteiligter eine Nachricht erhalten hat, diese nicht abstreiten darf. Infolgedessen darf der andere Beteiligte nicht verleugnen darf die Nachricht versendet zu haben oder eine Datei gelöscht zu haben.


\paragraph{Zurechenbarkeit}
Die Zurechenbarkeit gibt an, dass eine gewisse Aktion oder ein bestimmte Entscheidung einer bestimmten Person die Verantwortlichkeit zugeordnet wird. Dieses Schutzziel ist nicht gegeben, wenn beispielsweise in einem System ein nicht realer Benutzer mit dem Namen "TestUserAdmin" existiert. Wenn dieses Konto für viele verschiedene Personen zugreifbar ist, kann nicht festgestellt werden, wer der Verantwortliche für die Aktionen dieses Accounts ist.


\section{EMS Informationssicherheit}


\newpage