\chapter{Grundlagen der Informationssicherheit}
\strahlhofer

%Reis seine erster Foliensatz
\section{Allgemeines zu Informationssicherheit}
\subsection{Definition}
Bei der Informationssicherheit geht es grundsätzlich darum Informationen zu schützen. Informationen spielen in der heutigen Welt eine wichtige Rolle. Es gibt zwei Möglichkeiten Informationen zu verarbeiten. Diese sind entweder in einer digitalen oder analogen Form anzufinden.
\\
Weiters wird überprüft, dass eine fremde, nicht befugte Person oder ein nicht befugtes Unternehmen, auf keinen Fall den Zugriff auf die unternehmensinternen Systeme und Benutzerdaten erhält. Ein Unternehmen muss dafür Maßnahmen setzen, um die Nutzer ihres Services zu schützen. Diese Maßnahmen sollen verschiedenste Bedrohungen vermeiden und zur Datensicherheit, IT-Sicherheit und Datensicherung beitragen.
\\

%Reis Folie
\section{Relevante Standards und Normen}

\subsection{Länder spezifische Standards und Normen}
Viele verschiedene Länder haben ihre eigenen Standards oder Nachschlagewerke, in denen Sie verschiedensten Unternehmen versuchen eine Hilfe zu bieten, zum Thema Management der Informationssicherheit.
Deutschland besitzt das BSI(Bundesamt für Sicherheit in der Informationstechnik) in dem man sich als Firma auch zertifizieren kann. Österreich stellt das Österreichische Informationssicherheithandbuch zu Verfügung. Die Schweiz verwendet hingegen MELANI (Melde- und Anlaufsstelle für Informationssicherheit).


%https://de.wikipedia.org/wiki/ISO/IEC-27000-Reihe
%https://www.onlinesicherheit.gv.at/Themen/Experteninformation/Normen-und-Standards/ISO-IEC-27000.html
\subsection{ISO/IEC-27000-Reihe}
Die ISO/IEC 27000-Reihe ist eine Ansammlung von Standards über das Informationssicherheitsmanagement (ISMS) und gibt Ratschläge über gewisse Best Practices in diesem Bereich. Sie wurden von der International Organization for Standardization (ISO) und der International Electronic Commission (IEC) publiziert. 
Diese Zertifizierungs-Reihe ist so absichtlich so breit beschrieben, da es für alle möglichen Firmen in jeglicher Größe und Form anwendbar ist.
Diese Praktiken sollen mittels Informationssicherheitskontrollen das Managen von Informationsrisiken vereinfachen. Die ISO/IEC 27000 Familie besitzt genau 58 Standards.

%Reis ITIL Foliensatz 1
%https://en.wikipedia.org/wiki/ITIL
\subsection{ITIL}
ITIL (Information Technology Infrastructure Library) ist eine Ansammlung von Best Practices. Diese kann von Unternehmen verwendet werden um das Service Management zu überwinden und zusätzlich zeigt diese die Möglichkeiten der heutigen Technologie auf. Im Jahr 2013 hat die Firma AXELOS die Besitzrechte erlangt.
Der neuste Stand dieser Praktik ist ITILv4, die am 18. Februar 2019 publiziert wurde.

%https://www.bsi.bund.de/SharedDocs/Downloads/DE/BSI/Grundschutz/Kompendium/Elementare_Gefaehrdungen.pdf?__blob=publicationFile&v=4
%Reis Foliensatz 1
\section{Bedrohungen im Bereich Informationssicherheit}
In der Informationssicherheit gibt es viel verschiedene Arten von Bedrohungen die sich laufend verändern. Im BSI IT-Grundschutz sind 47 unterschiedliche Arten von Bedrohungen der Informationssicherheit aufgelistet.

%bsi und folien von reis
\paragraph{Wichtige Bedrohungen in Betrachtung der Informationssicherheit bezogen auf EMS (Event Management Software)}
\begin{itemize}
	\item \textbf{Gefährdungen durch natürliche Ereignisse:} Darunter fallen Naturkatastrophen, Feuer, Wasser, Verschmutzungen durch Staub, Korrosion und ungünstige klimatische Bedingungen die zur Zerstörung der IT-Infrastruktur führen.
	\item \textbf{Ausfälle oder Störungen:} Diese können von der Stromversorgung, Kommunikationsnetzen, Versorgungsnetzen oder Dienstleistern kommen.
	\item \textbf{Schadprogramme:} darunterfallen beispielsweise Viren, Würmer, Trojaner, Adware, Spyware, Ransomware.
	\item \textbf{Verhinderungen von Diensten (Denial of Service):} Dies werden auch DoS-Angriffe genannt, versuchen durch eine hohe Anzahl an Anfragen Störungen von Geschäftsprozessen oder IT-Ausfälle herbeizuführen.
	\item \textbf{Social Engineering:} Hier gibt sich eine Person als Administrator der eigenen Firma aus. Dieser Angreifer versucht beispielsweise per Telefonat einen Mitarbeiter durch Leichtgläubigkeit dazu  sein Anmeldeinformationen auszugeben.
	\item \textbf{Identitätsdiebstahl:} Phishing beispielsweise ist eine Technik die Angreifer verwenden um Kundendaten zu erlangen. Es wird unter anderem eine Anmeldeseite erstellt die identisch zur eigenen ist, um Kunden im Glauben zu lassen, dass diese nichts falsch machen.
	\item \textbf{Ressourcenmangel:} Ein Ressourcenmangel kann auftreten wenn unzureichende oder alte Hard- oder Software verwendet wird. 
\end{itemize}

%folien
\section{Gründe für das Informationssicherheitsmanagement}
\subsubsection{Notwendigkeit}
Es liegt eine gesetzliche Notwendigkeit im Sinne des Treffens der geeigneten technischen und organisatorischen Maßnahmen vor. Beispielsweise die EU-DSGVO, das Telekommunikationsgesetz, das Urheberrechtsgesetz.

\subsubsection{Wettbewerbsvorteil}
Wenn eine Firma einen hohen Stand an Informationssicherheit besitzt, indiziert es damit das das Datensicherheitsniveau sehr gut gepflegt ist. Das Unternehmen kann deswegen konkurrenzfähig bleiben. Zusätzlich kann das eigene Unternehmen mit Kunden ins Geschäft kommen, die einen hohen Wert auf Informationssicherheit setzen. Diese können z.B. Banken sein die bestimmte Zertifikate voraussetzen.
Durch Zertifikate kann nachgewiesen werden, dass der Betrieb über ein hohes Sicherheitsniveau verfügt.

\subsubsection{Risiken identifizieren}
Ein Vorteil für ein Unternehmen ist, dass Risiken die mit den Bedrohungen der Informationssicherheit zu tun haben, eher erkannt werden und infolgedessen auch Schäden verhindert werden können.

%https://wotan-monitoring.com/de/news/news-detail/die-schutzziele-der-informationssicherheit-verfuegbarkeit-integritaet-und-vertraulichkeit.html#:~:text=Informationssicherheit%20und%20ISO%2FIEC%2027001&text=Und%20im%20Hauptdokument%2C%20der%20ISO,Vertraulichkeit%E2%80%9C%20und%20deren%20Aufrechterhaltung%20definiert.
%Folien reis
%https://www.bsi.bund.de/SharedDocs/Downloads/DE/BSI/Grundschutz/Kompendium/IT_Grundschutz_Kompendium_Edition2020.pdf%3F__blob%3DpublicationFile%26v%3D6
\section{ISO/IEC 27001}
Die ISO/IEC 27001 definiert sogenannte Schutzziele oder Anforderungen an ein Informationsmanagementsicherheitssystem (ISMS). In dem Dokument wird genau auf die Einhaltung der Bereiche Vertraulichkeit(confidentiality), Integrität(integrity) und Verfügbarkeit(availability) eingegangen. Laut ISO/IEC 27000, gibt es weitere Faktoren, die auch ein Teil der Informationssicherheit sein können. Darunter zählen die Authentizität(authenticity), die Zurechenbarkeit(accountability), die Nicht-Abstreitbarkeit(non-repudiation) und die Verlässlichkeit(reliability).

\subsection{Schutzziele der ISO/IEC 27001}
%Reis Folie
\subsubsection{Vertaulichkeit}
Die Vertraulichkeit beschreibt die Eigenschaft, dass eine Information für eine nicht autorisierte Person oder ein Unternehmen zugänglich gemacht wird. Somit ist der Schutz gegen Verrat oder Spionage gewährleistet. Darunter wird verstanden, dass Kennzahlen, wie Umsatzzahlen nicht an die Öffentlichkeit gelangen. 
Der Fremdzugriff auf Daten wird durch die Versendung von verschlüsselten Tokens verhindert. Somit kann auf keinen Benutzer rückgeschlossen werden.

%https://www.bsi.bund.de/SharedDocs/Downloads/DE/BSI/Grundschutz/Kompendium/IT_Grundschutz_Kompendium_Edition2020.pdf%3F__blob%3DpublicationFile%26v%3D6
\subsubsection{Integrität}
Die Integrität laut dem BSI beschäftigt sich mit der Korrektheit von Informationen. In der Informationssicherheit wird damit assoziiert, dass eine unbefugte Person keine Daten manipulieren kann und darf. Darunter wird verstanden, dass Veränderungen wie Datensätze einfügen, verändern oder löschen nicht möglich ist.
Im Rahmen von Softwaretests kann überprüft werden ob Daten von befugten Individuen manipuliert werden. Dieser Aspekt kann beispielsweise mittels Hashfunktion gewährleistet werden.
Dieses Schutzziel ist gegeben, da ein Angreifer nicht wissen kann, welchen Token ein Promoter besitzt.

%Reis Folie
\subsubsection{Verfügbarkeit}
Unter Verfügbarkeit wird in diesem Schutzziel verstanden, dass ein IT-System oder eine Anwendung, für den Benutzer, welcher ein Mitarbeiter oder ein Kunde sein, jederzeit zu Verfügung steht. Fälle wie z.B. der Ausfall oder die nicht Verfügbarkeit durch eine DDoS-Attacke (Distributed Denial of Service) oder eine DoS-Attacke(Denial of Service) eines Bank-Systems. Wenn das System nur teilweise oder gar nicht funktioniert, können möglicherweise keine Überweisungen getätigt werden und folge dessen sind die Angestellten und Konsumenten eingeschränkt. 
Rein theoretisch läuft die Software dauerhaft, da sie auf dem Cloud-Service AWS (Amazon Web Services) deployed ist. Praktisch gesehen würden Kosten entstehen die nicht abgedeckt werden können, da der Auftraggeber der Software momentan keinen Nutzen in der Betreibung der App sieht. Da durch die Covid-19 Pandemie keine Events stattfinden können. Weiters muss jeder Kontakt zu anderen Personen vermieden werden. 

%https://en.wikipedia.org/wiki/Man-in-the-middle_attack
\subsubsection{Authentizität}
Bei diesem Schutzziel geht es darum, dass ein System gewährleistet, das ein Kommunikationspartner wirklich die Person ist, die er vorgibt zu sein. Bei der Überprüfung der Authentizität muss ein Benutzer seine Identität vorerst bestätigen. Das kann durch verschiedene Methoden erfolgen, wie die Eingabe von Benutzername und Passwort oder die Sendung des Reisepasses oder eine Anmeldung via Fingerabdruck oder Gesichtserkennung. Danach kann durch ein Kontrollmodul die Identitätskontrolle an der Person durchgeführt werden und dadurch authentifiziert werden. Am Ende dieses Vorgangs ist eine Person im System angemeldet und es werden im Rahmen der Autorisierung bestimmte Rollen oder Rechte zugewiesen.
Dieses Schutzziel kann \textbf{Man-in-the-Middle-Angriffe} verhindern. Bei dieser Cyberattacke setzt sich eine Person zwischen die Kommunikation zweier verschiedener Parteien und kann damit Transaktionen manipulieren. 
Eine Person kann sich in unserem System durch die Eingabe eines Benutzernamens und Passworts anmelden. Die Informationen werden an den Server gesendet und dort überprüft. Der Server versendet einen mit einem Geheimschlüssel codierten Token an den Benutzer zurück. Der Fremdzugriff ist aus diesem Grund verwährt, da dieser die Gültigkeit erkennen kann.

%https://de.wikipedia.org/wiki/Informationssicherheit#Zugangskontrolle
\subsubsection{Zugriffssteurung}
Als nächstes Schutzziel gibt es die Zugriffssteuerung. Diese wird gewährleistet, wenn der Zugang zu Informationen oder Ressourcen nur autorisiert statt findet. Somit wird ein Berechtigungskonzept umgesetzt. 
Dieses Schutzziel wird nach dem sich ein Benutzer angemeldet hatt durchgeführt. Hier wird bestimmt per verschlüsselten Token, ob es sich um einen Promoter oder einen Administrator handelt. Es existiert für diese Verschlüsselung zwei verschiedene Geheimschlüssel.

%Reis Folie
\subsubsection{Verlässlichkeit}
Bei der Verlässlichkeit geht es darum, dass ein System immer konsistent und nach den festgelegten Spezifikationen läuft. Darüber hinaus soll es zu jeder Zeit folgerichtige Ergebnisse liefern. Beispielsweise wenn ein System welches als Hauptfunktionalität das Verschlüsseln von Passwörtern hat. Hierbei darf es nicht dazu kommen, dass eine inkonsistente Verschlüsselung bei jedem zehnten Benutzer verwendet wird.
In der Software hat jede Verschlüsselung nur einen bestimmten Geheimschlüssel gibt. Des weiteren ist nur der Server davon 

%https://en.wikipedia.org/wiki/Information_security#Non-repudiation
\subsubsection{Nichtabstreitbarkeit/Verbindlichkeiten}
Dieses Schutzziel gibt an, dass eine Person, die ihren Verpflichtungen und Aufgaben aus einem im Vorhinein festgelegten Vertrag nach geht. Des weiteren gilt, dass wenn ein Beteiligter eine Nachricht erhalten hat, diese nicht abstreiten darf. Infolgedessen darf der andere Beteiligte nicht verleugnen die Nachricht versendet zu haben oder z.B. die Mail in der die Passwortzurücksetzung erfolgt gelöscht zu haben.

\subsubsection{Zurechenbarkeit}
Die Zurechenbarkeit gibt an, dass zu einer gewissen Aktion oder einer bestimmten Entscheidung, die Verantwortung einer bestimmten Person zugeordnet werden kann. Dieses Schutzziel ist nicht gegeben, wenn beispielsweise in einem System ein nicht realer Benutzer mit dem Namen TestUserAdmin existiert. Wenn dieses Konto für viele verschiedene Personen zugreifbar ist, kann nicht festgestellt werden, wer der Verantwortliche für die Aktionen dieses Accounts ist.
%jo Datensätze in MongoDB ausbessern
%Screenshot maybe

\section{EMS (Event Management Software) Informationssicherheit}
Bei der Erstellung eines Anmeldesystems gibt es sehr viele Aspekte die beachtet werden müssen das Benutzer keine Komplikationen mit dem System besitzen. Um sicherzustellen das Informationen die das Vorhaben umfassen, geschützt bleiben. Die Schutzziele laut ISO/IEC 27001 wurden hier berücksichtigt. Unser Auftraggeber wird sich nicht für den ISO/IEC 27000 Standard zertifizieren lassen, da die Kosten nicht gedeckt werden können.
Jeder User ist durch die Versendung eines verschlüsselten Tokens gesichert. Dieser Token kann nicht von unautorisierten Personen oder Unternehmen verwendet werden um Informationen über einen Benutzer der EMS-Software herauszufinden. 

\newpage