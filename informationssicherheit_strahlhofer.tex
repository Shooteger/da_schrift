\chapter{Grundlagen der Informationssicherheit}
\strahlhofer

%Reis seine erster Foliensatz
\section{Allgemeines zu Informationssicherheit}
\subsection{Definition}
Bei der Informationssicherheit geht es grundsätzlich darum Informationen zu schützen. Informationen spielen in der heutigen Welt eine wichtige Rolle. Die zwei Arten Informationen zu verarbeiten sind entweder in einer digitalen oder analogen Form. 
\\
Weiters wird überprüft, dass eine fremde, nicht befugte Person oder ein nicht befugtes Unternehmen, auf keinen Fall den Zugriff auf die unternehmensinternen Systeme und Benutzerdaten erhält. Ein Unternehmen muss dafür Maßnahmen setzen, um die Nutzer ihres Services zu schützen. Diese Maßnahmen sollen verschiedenste Bedrohungen vermeiden und zur Datensicherheit, IT-Sicherheit und Datensicherung beitragen.
\\

%https://www.bsi.bund.de/SharedDocs/Downloads/DE/BSI/Grundschutz/Kompendium/Elementare_Gefaehrdungen.pdf?__blob=publicationFile&v=4
%Reis Foliensatz 1
\section{Bedrohungen für Informationssicherheit}
In der Informationssicherheit gibt es viel verschiedene Arten von Bedrohungen die sich laufend verändern. Im BSI sind 47 unterschiedliche Arten von Bedrohungen in der Informationssicherheit aufgelistet.

\begin{itemize}
	\item \textbf{Gefährdungen durch natürliche Ereignisse} darunter fallen Naturkatastrophen, Feuer, Wasser, Verschmutzungen durch Staub, Korrosion und ungünstige klimatische Bedingungen die zur Zerstörung der IT-Infrastruktur führen.
	\item \textbf{Ausfälle oder Störungen} diese können von der Stromversorgung, Kommunikationsnetzen, Versorgungsnetzen oder Dienstleistern kommen
	\item \textbf{Schadprogramme} beispielsweise Viren, Würmer, Trojaner, Adware, Spyware, Ransomware
	\item \textbf{Verhinderungen von Diensten (Denial of Service)} auch DoS-Angriffe genannt, versuchen Störungen von Geschäftsprozessen oder IT-Ausfälle herbeizuführen.
	\item \textbf{Social Engineering, hier gibt sich eine Person als Administrator der eigenen Firma aus. Dieser Angreifer versucht beispielsweise per Telefonat einen Mitarbeiter durch Leichtgläubigkeit dazu  sein Anmeldeinformationen auszugeben.
	\item \textbf{Identitätsdiebstahl} beispielsweise Phishing ist eine Technik die Angreifer verwenden um Kundendaten zu erlangen. Es wird unter anderem eine Anmeldeseite erstellt die identisch zur eigenen ist, um Kunden im Glauben zu lassen, dass sie nichts falsch machen.
	\textbf{Ressourcenmangel} kann auftreten wenn alte Hard- oder Software verwendet wird. 
\end{itemize}


\section{Gründe für das Informationssicherheitsmanagement}
\paragraph{Gesetzliche Notwendigkeit}
Es liegt eine gesetzliche Notwendigkeit im Sinne des Treffens der geeigneten technischen und organisatorischen Maßnahmen vor beispielsweise die EU-DSGVO, das Telekommunikationsgesetz, das Urheberrechtsgesetz.

\paragraph{Wettbewerbsvorteil}
Wenn eine Firma einen hohen Stand an Informationssicherheit besitzt, kann sie Konkurrenzfähig bleiben. Zusätzlich kann man mit Partnerunternehmen ins Geschäft kommen die einen hohen Wert auf Informationssicherheit setzen. Durch Zertifikate kann man Nachweisen das der Betrieb über ein hohes Sicherheitsniveau verfügt.

\paragraph{Risiken identifizieren}
Ein Vorteil für ein Unternehmen ist das Risiken die mit den Bedrohungen der Informationssicherheit zu tun habe, eher erkannt werden und infolgedessen auch Schäden umgangen werden können.

\newpage