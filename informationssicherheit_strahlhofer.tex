\chapter{Grundlagen der Informationssicherheit}
\strahlhofer

%Reis seine erster Foliensatz
\section{Allgemeines zu Informationssicherheit}
\subsection{Definition}
Bei der Informationssicherheit geht es grundsätzlich darum Informationen zu schützen. Informationen spielen in der heutigen Welt eine wichtige Rolle. Die zwei Arten Informationen zu verarbeiten sind entweder in einer digitalen oder analogen Form. 
\\
Weiters wird überprüft, dass eine fremde, nicht befugte Person oder ein nicht befugtes Unternehmen, auf keinen Fall den Zugriff auf die unternehmensinternen Systeme und Benutzerdaten erhält. Ein Unternehmen muss dafür Maßnahmen setzen, um die Nutzer ihres Services zu schützen. Diese Maßnahmen sollen verschiedenste Bedrohungen vermeiden und zur Datensicherheit, IT-Sicherheit und Datensicherung beitragen.
\\


%Systemplanung und Projekte Entwicklung von Felix Schwab, Ingridschwab-Matkovits
\subsection{Risikomanagement im Bereich Informationssicherheit}
Wenn in einem Unternehmen die IT-Infrastruktur ausgebaut wird, müssen Risiken vermieden oder vermindert werden. Wenn ein Unternehmen, das viele private oder sensible Daten verarbeitet, sollte jedenfalls eine Risikoanalyse zu der IT-Sicherheit durch geführt werden. Durch dieses Verfahren kann man zukünftige Bedrohungen abschätzen und sich dazu Bewältigungsstrategien überlegen.

\subsubsection{Risiko}
Unter dem Begriff Risiko assoziiert man im Bereich IT-Sicherheit die Schadenshöhe, wenn ein bestimmtes Ereignis eintritt. Diese Ereignisse haben einen negativen Effekt auf das Unternehmen. Sie tragen dazu bei, dass materielle oder immaterielle Schäden auftreten können oder sogar Arbeitsprozesse abstoppen.

\subsubsection{Maßnahme}
Eine Maßnahme ist eine Handlung, die es ermöglicht ein Risiko zu umgehen oder es einzugrenzen, hier spricht man von der Schadensminimierung. 
\\
\\
\subsection{Ablauf zur Findung von Maßnahmen im Risikomanagement}
\begin{enumerate}
    \item Risiken identifizieren
    \item Risiken analysieren
    \item Risiken bewerten
    \item Risiken vermeiden
\end{enumerate}

%https://www.business-wissen.de/hb/risiken-identifizieren/#:~:text=Insofern%20ist%20die%20Identifikation%20von,Sch%C3%A4den%20unvorhergesehen%20und%20%C3%BCberraschend%20eintreten.
\subsubsection{Risikoidentifikation}
Alle Risiken zu identifizieren die möglicherweise auftreten können, kann sich als sehr schwer erweisen. Es können etwaige Risiken auftreten, von denen noch nichts bekannt war oder die man nicht in Betracht gezogen hat. Aus diesen Gründen ist die Identifikationsphase eine sehr wichtige. Sie muss sorgfältig durchgeführt werden und mit verschiedenen Ansätzen abgewickelt werden. Fortlaufend sollte deswegen regelmäßig überprüft werden ob sich Risiken verändern oder möglicherweise hinzukommen.
\\

%Methoden gefunden: Im BUCH und auf
%https://de.wikipedia.org/wiki/Risikoidentifikation#:~:text=Als%20Methoden%20kommen%20f%C3%BCr%20bestehende,oder%20die%20Delphi%2DMethode%20ermitteln.


\paragraph{Bestehende Risiken und Potenzielle Risiken}
Die Risikoidentifikation beschäftigt sich mit zwei Arten von Risiken einmal die bekannten Risiken und die potenziellen Risiken. Folgende Methoden können zur Bewältigung angewandt werden:
\begin{enumerate}
    \item \textbf{Methoden um bestehenden/bekannte Risiken zu identifizieren}
    \begin{itemize}
        \item Einsatz von Checklisten
        \item SWOT-Analyse
        \item Befragungen von erfahrenen Mitarbeitern
        \item Befragungen von Experten mit spezifischen Fachwissen
    \end{itemize}
    \item \textbf{Methoden um Potenzielle Risiken zu identifizieren}
    \begin{itemize}
    	\item Brainstorming
    	\item Delphi-Methode
    	\item Mindmapping
    	\item Morphologischer Kasten
    	\item Brainwriting (bzw. 6-3-5 Methode)
    \end{itemize}
\end{enumerate}

%https://de.wikipedia.org/wiki/Risikoidentifikation#Checklisten_und_Mitarbeiterbefragungen
\paragraph{Einsatz von Checklisten}
Bei dieser Methode werden standardisierte Checklisten verwendet. Diese sollen einem Unternehmen aufzeigen welche Risiken zu beachten sind, wenn es um die Informationssicherheit geht. Es besteht nur ein Nachteil, es werden nicht alle für das eigene Unternehmen relevanten Risiken genannt.

%https://smallbusiness.chron.com/security-swot-analysis-40526.html
\paragraph{SWOT-Analyse}
SWOT(Strengths, Weaknesses, Opportunities, Threats) ist eine Methode verschiedene Faktoren zu analysieren. Durch infrage stellen der Stärken (Strengths) und Schwächen (Weaknesses) kann man unternehmensinterne Faktoren erkennen. Durch hinterfragen von Möglichkeiten (Opportunities) und Bedrohungen (Threats) findet man externe Faktoren die mit einfließen.

%https://de.wikipedia.org/wiki/Risikoidentifikation#Checklisten_und_Mitarbeiterbefragungen
\paragraph{Befragungen}
Bei diesen Umfragen werden entweder erfahrene Mitarbeiter oder Experten mit gewissen Fachwissen befragt. Mitarbeiter liefern mehr interne Unternehmensrisiken. Experten finden eher externe unternehmensbedrohliche Risiken. Befragungen können in einer digitalen, schriftlichen oder mündlichen Arten statt finden.

%Buch S. 38
\paragraph{Brainstorming}
Hier versammeln sich eine Gruppe von unternehmensinternen Personen bestehend aus bevorzugter Weise aus fünf bis sieben Personen. 
Beim Brainstorming Prozess werden Vorschläge und Ideen nur verbal geäußert und danach dokumentiert. Eine Person der sogenannte Moderator leitet diesen Kommunikationsprozess. Vorzugsweise sollte dieser Vorgang nur zehn bis maximal zwanzig Minuten andauern.
Die Bewertung der erhobenen Ideen / Vorschlägen darf nicht gleichzeitig mit der Erhebung ablaufen und dürfen deswegen nur in einer zweiten Phase abgehalten werden.

Es gibt gewisse Regeln die zu beachten sind:
\begin{itemize}
	\item Keine Barrieren, d.h. man kann alles Vorschlagen, auch nicht umsetzbare oder absurde Ideen. Keine Idee ist zu extrem.
	\item Keine Diskriminierung, damit ist gemeint, dass kein Teilnehmer für eine geäußerte Idee kritisiert werden soll. Diese Regel ist sehr wichtig, da es sonst den kreativen Prozess stört oder sogar verhindert.
\end{itemize}

%Buch S. 38
\paragraph{Brainwriting (6-3-5 Methode)}
Bei dieser Methode gibt es sechs Personen die sich zusammensetzen, jeder Teilnehmer soll dann jeweils 3 Ideen schriftlich festhalten, in einer Zeit von ungefähr 5 Minuten.
Als nächster Schritt wird dann das Schriftstück in der Runde verteilt, dass kann entweder in eine Richtung weiter gegeben werden oder sie werden durchgemischt und dann verteilt. 
Als Ergebnis dieser Methode sollten nach 30 Minuten 108 Ideen aufgeschrieben worden sein.
Der Vorteil dieser Methode ist, dass alle Ideen dokumentiert worden sind und die Ergebnisse deshalb in einer schriftlichen Form vorhanden sind.

%Buch S. 41
\paragraph{Delphi-Methode}
Bei der Delphi-Methode wird eine Problemstellung oder ein Risiko von einer Gruppe aus verschiedenen und nicht voneinander abhängigen Experten bzw. Fachleuten bearbeitet. 
Jeder Teilnehmer erarbeitet einen Lösungsansatz zu dem gestellten Risiko auf einem Schriftstück.
Die Ausarbeitungen werden eingesammelt und anonymisiert weitergegeben.
Der erhaltene Lösungsansatz soll jetzt kritisiert werden, d.h. falls die Person Denkfehler oder falsche Ansätze verwendet. Aus dieser Bearbeitung, soll man für das eigene Konzept Ideen oder Änderungen finden und integrieren.
Dieser Prozess soll mehrere male wiederholt werden, so dass die Gruppe eine gemeinsame Lösung beschließt.
Die Anonymität spielt bei der Methode eine große Rolle und trägt dazu bei, dass jeder Teilnehmer kreative und alternative Standpunkte einnimmt.
Der Vorteil diese Methode anzuwenden ist, dass unabhängige Fachleute eingesetzt werden und diese über einen größeren Zeitraum an dieser Problematik arbeiten.
Ein daraus folgender Nachteil ist ein hoher Zeit- und Organisationsaufwand.


\paragraph{Mindmapping}


\paragraph{Morphologischer Kasten}





%https://www.projektmagazin.de/methoden/risikomatrix

\newpage
%https://www.link11.com/de/blog/it-sicherheit/it-risiken-bedrohungen-kennen-bewerten/#:~:text=Zu%20den%20gef%C3%A4hrlichsten%20externen%20Bedrohungen%20z%C3%A4hlen%20gezielte%20oder%20willk%C3%BCrliche%20Cyber%2DAttacken.&text=Die%20h%C3%A4ufigsten%20Bedrohungen%20in%20der,Advanced%20Persistent%20Threats
%Reis sein erster Foliensatz
\subsection{Bedrohungen im Bereich IT-Sicherheit}
\subsubsection{Cyber Angriffe}
So genannte Cyber Angriffe können für ein Unternehmen zu einem starken Problem führen.

\newpage