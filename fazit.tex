\chapter{Fazit}

Zusammenfassend kann gesagt werden, dass das Projekt im Großen gesehen gut verlaufen ist und fast jeder seine Abgabetermine eingehalten hat.

Viele Kleinigkeiten verlangsamten die Entwicklungen. Dies begann schon zu Anfang der Schuljahres mit der Wahl des Vorgehensmodelles und des Ticketsystems.
Für das Projekt wurde im vorherigen Schuljahr schon ein Projektantrag verfasst. In diesem war angegeben, dass das Vorgehensmodell RUP und das Issue-Tracking
System Jira genutzt werden soll. In den Sommerferien vor Beginn der Abschlussklasse wurde sich dementsprechend im Team, was sich zu dem Zeitpunkt noch
aus fünf Personen zusammensetzte, begonnen in diesen Gebieten Wissen anzusammeln. Leider wurden wir erst nach der Annahme unseres Antrags auf interne Abteilungsrichtlienien
hingewiesen. Wir mussten das komplette Kapitel \textbf{Vorgehensmodell} auf SCRUM umschreiben und auch das von der Abteilung der Schule vorgegebene Programm \textbf{Vivify} benutzen.

Dies machte die geleistet Vorarbeit im Programm Jira und RUP in den Sommermonaten zunichte. Zwei Monate später musste auch noch ein Teammitglied, aufgrund von nicht näher beschriebenen
Komplikationen aus dem Team austreten. Aus organisatischer Sicht waren die ersten zwei Monate der Entwicklungsphase dementsprechend nicht effizient geplant.

Weiters gab es Komplikationen mit dem zuerst geplanten Chatsystem auf Basis von Sockets und dem Ticketsystem. Nach mehreren erfolglosen Versuchen des gesamten Teams am Chatsystem,
wurde intern entschieden, diesen gesamten Teil der Arbeit neu zu planen und entwerfen. Jetzt exisitert ein funktionierender Chat auf Basis von Angular und Firebase, welcher nur noch
in EMS inkludiert werden muss.

Das Ticketsystem\dots

Das gesamte Team hat durch dieses Projekt einiges gelernt was in zukünftigen Projekten bedacht wird.
In der Softwareentwicklung ist die Organisation das A und O zusammen mit dem Vorgehensmodell. Hier muss vor beginn der Entwicklung alles fertig geplant sein,
eine Änderung an dieser Stelle bringt viele Nachteile und Komplikationen mit sich.
Weiters wird in Zukunft, falls das Vorgehensmodell SCRUM verwendet wird und eine User-Story nach zwei maligem Verschieben noch immer nicht Implementiert ist,
dises Story einem internen Review im Team unterzogen und besprochen, wie mit dieser weiter verfahren werden soll.

In Zukunft sind noch mehrer Updates und Patches von EMS eingeplant, da dieses System im realen Betrieb genutzt werden soll. Bis Ende Juli 2021 wird der fertige Chat vollkommen
in EMS integriert sein. Ebenso zu selbigem Datum wird das Ticketsystem voll funktionsfähig sein. Damit ist die App in den wichtigen Sommermonaten Juli und August, bereit für
die Nutzung unter realen Bedingungen.

Das Projekt brachte jedem Teammitglied viel neues, vor allem praxisbezogenes, Wissen. Die Harmonie untereinander im Team war stehts hoch, wir würden jederzeit gerne wieder ein
Softwareprojekt zusammen beginnen. (Außer Benni)