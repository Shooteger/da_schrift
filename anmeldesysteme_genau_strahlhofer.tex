\chapter{Anmeldesysteme}

%https://en.wikipedia.org/wiki/Authorization
\section{Authorization}
Authorization hat als Funktionalität die Vergabe von bestimmten Rechten und Rollen zu bestimmten Ressourcen. Dieses Thema gehört der Informationssicherheit an und befasst sich mit der Zugriffskontrolle. Bei der Authorization wird ein für das Unternehmen ein Zugriffsmechanismus implementiert. Wenn z.B. Mitarbeiter sich in einem von der Firma erstellten System anmeldet, werden diesem / dieser Mitarbeiter / in bestimmte Rollen und Rechte zugewiesen. Dies kann bedeuten das der Mitarbeiter gewissen Zugriff besitzt eine Datei im System zu erstellen, diese zu verändern oder möglicherweise diese zu löschen.

\subsubsection{Rollen}
\begin{itemize}
	\item \textbf{Administratoren:} Diese Rolle ist eine der wichtigsten in einem System. Dieser kann Mitarbeiter-Konten in einem System erstellen, falls diese neu in der Firma sind. Er kann eine Löschung durchführen, wenn ein Angestellter kündigt oder gekündigt wird. Diese Tätigkeit ist unbedingt notwendig und muss sofort passieren. Es kann sonst ein Risiko darstellen, da der ehemalige Mitarbeiter / in sonst weiter im System Informationen manipulieren könnte. Des weiteren kann er Details von einem Benutzer verändern, wenn beispielsweise diese / r heiratet und einen neuen Nachnamen besitzt. 
\end{itemize}