\chapter{Aspekte eines Anmeldesystemes und Biometrie}
\strahlhofer

\section{Prozesse des Anmeldevorgangs}
\begin{center}
\begin{figure}[h]
    \centering
    \includegraphics[width=9cm]{authentisieren-authentifizieren.png}
    \caption{Ablauf eines Anmeldevorgangs (Abbildung von wikipedia)}
\end{figure}
\end{center}

%https://www.dr-datenschutz.de/authentisierung-authentifizierung-und-autorisierung/
\subsection{Authentisierung}
Bei der Authentisierung muss von einer Person ein Nachweis gegeben werden, dass sie die Person ist die sie behauptet zu sein. Dieser Nachweis kann durch drei verschiedene Methoden nachgewiesen werden, diese sind:
\begin{itemize}
	\item Die Person besitzt Wissen über eine Information, die nur der Kontoinhaber wissen kann, z.B. das Passwort oder Sicherheitsfragen sein.
	\item Das Individuum das sich versucht anzumelden und sendet beispielsweise einen Reisepass oder Führerschein als Konformation.
	\item Es wird vom Nutzer eine Information mitgesendet die er oder sie nur von sich selbst senden kann. Ein Beispiel wäre die Sendung von einem biometrischen Nachweis, dass kann entweder ein Fingerabdruck oder ein Gesichtsscan sein.
\end{itemize}

%https://www.dr-datenschutz.de/authentisierung-authentifizierung-und-autorisierung/
\subsection{Authentifizierung}
Bei diesem Schritt des Anmeldeablaufs, wird eine Überprüfung durchgeführt. Diese Prüfung soll analysiert die Informationen, die bei der Authentisierung erfasst worden sind. Können diese dem Zielkonto zugewiesen werden, dann ist der Nutzer die Person die er angibt zu sein.

%https://www.dr-datenschutz.de/authentisierung-authentifizierung-und-autorisierung/
\subsection{Autorisierung}
Die Autorisierung hat als Funktionalität die Bestätigung von bestimmten Rechten und Rollen zu bestimmten Ressourcen. Dieses Thema gehört der Informationssicherheit an und befasst sich mit der Zugriffskontrolle. Bei der Autorisierung wird für das Unternehmen ein Zugriffsmechanismus implementiert. Wenn z.B. ein Promoter sich in einem von der Firma erstellten System anmeldet, werden bestimmte Rollen und Rechte zugewiesen. Dies bedeutet das der Mitarbeiter nur einen bestimmten Zugriff auf gewisse Funktionalität besitzt.

\subsubsection{Rollen}
Bei der Anmeldung in die EMS-Software soll zwischen den Rollen Promoter und Administrator unterschieden werden. 
\begin{itemize}
	\item \textbf{Administratoren:} Die Rolle ist eine der wichtigsten in der Software. Dieser kann Benutzer und Events in einem System erstellen, verändern und löschen. Benutzer können zusätzlich deaktiviert werden, da diese entweder zu häufig falsche Anmeldeinformationen eingegeben haben oder weil die Person eine Anfrage darauf gestellt hat.
	\item \textbf{Promoter:} Diese Rolle sind Benutzer der Software, die Karten anfordern und verkaufen können. Sie können, beispielsweise die Funktionalitäten wie die Benutzer- und Eventverwaltung nicht bedienen.
\end{itemize}

%https://www.kaspersky.de/resource-center/definitions/biometrics
%Biometrie
\section{Biometrie}
%https://en.wikipedia.org/wiki/Biometrics
\subsection{Definition}
\begin{center}
	\textit{Biometrics are body measurements and calculations related to human characteristics}
\end{center}

Bei dieser Wissenschaft, schließt eine biometrische Identifikation nur auf eine bestimmte Person.
Durch diesen Aspekt ist die Biometrie eine sehr angesehne Technik in der Identifikation von Menschen.

\subsection{Methoden der Biometrie}
Eine biometrische Information kann durch viele unterschiedliche Methoden erhoben werden.
\begin{itemize}
	%https://www.livescience.com/62690-how-dna-ancestry-23andme-tests-work.html
	%https://www.ibia.org/biometrics-and-identity/biometric-technologies/dna#:~:text=DNA%20Biometrics,often%20in%20forensics%20and%20healthcare.&text=A%20feature%20of%20DNA%20identification,familial%20relationships%20via%20DNA%20testing.
	\item \textbf{DNA-Tests:} Ein DNA-Test wird in den unterschiedlichsten Arbeitsbereichen verwendet, z.B. in der Fornesik oder von der Polizei.
	Vorteile die DNA als biometrisches Mittel zu erwägen:
	\begin{itemize}
		\item Mit der DNA kann von einer unbekannten Person auf dessen Verwandten schließen. Dies ist die einzige biometrische Methode die diese Möglichkeit bietet.
		\item DNA Spuren und Fingerabdrücke können bei Tatorten gefunden werden. Diese Methode hilft Ermittlern schneller die Identität einer Person herauszufinden.
	\end{itemize}
	%https://de.wikipedia.org/wiki/Fingerabdruck
	\item \textbf{Fingerabdruck:} Die Fingerabdrücke sind bei jedem Menschen unterschiedlich. 
	Es gibt bis zum heutigen Tag keine bewissenen Fingerabdruckstest der auf zwei verschiedene Menschen schließen konnte. 
	Nicht einmal eineiige Zwillinge besitzen den gleichen Abdruck.
	Dieser Aspekt macht diese Methode so genau, es wird hier von einer Einzigartigkeit gesprochen.
	\paragraph{Vorteile}
	\begin{itemize}
		\item Wie schon beschrieben werden Fingerabdrücke von Ermittlern verwendet, um die Identität von Opfern oder verdächtigen Personen aufzudecken.
		\item Der Fingerabdruck wird bei den meisten Smartphones in der heutigen Zeit per eingebauten Scanner ermittelt. Diese biometrischen Informationen werden erhoben um das Gerät einer bestimmten Person oder auch Personen zuzuordnen.
	\end{itemize}
	\paragraph{Nachteile}
	\begin{itemize}
		\item In dem Fall, dass scih eine Person auf dem Finger verletzt, kann es dazu führen das der Fingerabdruck sich verändert. Aus diesem Grund kann sich die Person möglicherweise nicht mehr identifizieren.
	\end{itemize}
	%https://en.wikipedia.org/wiki/Facial_recognition_system
	%https://www.pcs.com/wissens-werte/biometrie/gesichtserkennung-von-angesicht-zu-angesicht#:~:text=Hohe%20Sicherheit%20ist%20mit%20Gesichtserkennung,in%20der%20Regel%20nicht%20erkannt.
	\item \textbf{Gesichtserkennung:} Hier wird ein Foto oder Video analysiert und überprüft ob diese Person mit den schon aufgenommenen Bildern in der Dantebank übereinstimmen. Hier werden Aspekte wir Abstand von Augen-Nase-Mund festgestellt.
	\paragraph{Vorteile}
	\begin{itemize}
		\item Eine Person hat keinen oder kaum einen Aufwand um sich bei dieser Methode zu identifizieren. Bei Smartphones muss man lediglich in die Frontal Kamera schauen.
		\item Gesichtserkennungssysteme besitzen den Vorteil mehrere Personen aufeinmal zu erkennen. Beispielsweise in einem Labor dürfen nur bestimmte Personen zutreten. Wenn zwei Personen nun gleichzeitig eintreten wollen und einer die Berechtigung dazu nicht besitzt, wird ein Alarm ausgelöst.
	\end{itemize}
	\paragraph{Nachteile}
	\begin{itemize}
		\item Eineiige Zwillinge können bei dieser Methode schwer unterschieden werden, da diese über ein fast identisches Gesicht besitzen.
	\end{itemize}
\end{itemize}

%https://candytech.in/10-worthy-fingerprint-scanner-smartphones-in-india/#:~:text=Motorola%20was%20the%20first%20company,fingerprint%20scanner%20at%20the%20top.
%https://en.wikipedia.org/wiki/Android_(operating_system)
\subsection{Biometrie auf Android Geräten}
Android ist eine Betriebssystem für Smartphones und Tablets. Der Erfolg des Betriebssystems, zeichnet sich damit aus, dass es Open Source ist. Das bedeutet ein Unternehmen kann Android frei und ohne Kosten nutzen.
\paragraph{Fingerabdruckssensoren}
Motorola hatte im Jahre 2011 das Motorola Atrix veröffentlicht. Dies war das erste Smartphone auf dem Weltmarkt, dass einen Fingerabdruckssensor eingebaut hatte.
Dieses Gerät bekam durch dieses Feature einen sehr hohen Bekanntheitsgrad und löste damit gleichzeitig einen Technologie Trend aus.
Im laufe der danachfolgenden Jahre baute Apple dann auch einen Fingerprintsensor in ihre Geräte ein. Des weiteren begannen dann andere Android Smartphone Hersteller auch diesen Sensor zu integrieren.
Über die Jahre sind viele verschiedene Arten von Sensoren entwicklet worden. Wie z.B.
\begin{itemize}
	\item Berühren eines Knopfes auf dem Gerät (z.B. Apple Geräte).
	\item Wischen über einen Knopf (z.B. Samsung S7 Edge).
	\item In-Display, den Finger auf das Display legen (z.B. Oneplus 8). 
\end{itemize}
\paragraph{Gesichtserkennung}
Die Gesichtserkennung gibt es schon für die meisten Android Geräte. Diese sind häufig sehr unsicher, es exisitieren viel Fälle wo das Feature geknackt worden ist. Aus diesem Grund kennzeichnen die meisten Hersteller diese Methode als unsicher. 
\\
%https://www.kiroku-just-write.de/2020/09/01/gesichtserkennung/#:~:text=Die%20grunds%C3%A4tzliche%20Voraussetzung%20f%C3%BCr%20funktionierende,f%C3%BCr%20jeden%20Menschen%20einzigartig%20sind.
%https://www.consumentenbond.nl/veilig-internetten/gezichtsherkenning-te-hacken
\paragraph{consumentenbond}
Im Jahre 2019 wurde eine Studie von einer niederländischen Organisation zu dem Thema Gesichtserkennung bei Smartphones, durchgeführt. 
Diese haben 60 Smartphones von unterschiedlichen Herstellern getestet. 26 dieser Handys konnten mittels eines Portraits des Telefonbesitzers entsperrt werden.

%https://en.wikipedia.org/wiki/Touch_ID
\subsection{Biometrie auf iOS Geräten}
Apple Inc. ist eine Technologie Firma aus den Vereinigten Staaten, die durch den Verkauf von Computern in den siebziger Jahren. Das Unternehmen bringt jährlich Smartphones(iPhones), Tablets(iPads), Laptops(MacBooks) auf den weltweiten Markt.
\paragraph{Touch ID}
Nachdem 
Im Jahre 2012 kaufte Apple die Firma AuthenTec. Dieses Unternehmen spezialisierte sich auf das Lesen von Fingerabdrücken.
Im darauffolgendem Jahr brachte das Unternehmen dann das iPhone 5s heraus. Dieses Gerät hatte einen Fingerabdruckssensor im "Home-Button" eingebaut. Ab diesem Zeitpunkt konnten alle Benutzer zusätzlich zu ihrem Passwort eine sogenannte Touch ID zur Identifikation benutzen.
Die Fingerabdrücke wurde auf den Chips des Geräts gespeichert, anstatt in der Cloud. Das macht Angreifern es unmöglich die biometirschen Infomrationen abzufangen.
%https://www.pcwelt.de/international/Gesichtserkennung-auf-Android-Smartphones-einrichten-10824319.html
\paragraph{Face ID}
Das erste Smartphone von Apple, dass Gesichstserkennung besitzte, war das iPhone X(2017). Dieses Feature trug den Namen Face ID und löste größtenteils damit die Touch ID ab.
Alle Geräte die Face ID verwenden haben eine 3D-Sensoren die mehr Gesichtsmerkmale erkennen, als wie bei Android Geräten.


\subsection{Biometrie bei EMS}
Bei der Plannung der Software von EMS wurde ein Framework namens Ionic, das auf Angular aufbaut, verwendet. Dieses kann ein Projekt in zwei fast identisch aussehende Applikationen umwandeln, die auf Android und iOS laufen.
In der Dokumentation wurde Fingerprint AIO vorgestellt. Dieses Tool ist eine von Cordova bereitgestelltes Plugin, um einen Fingerabdruck auf iOS und Android zu verwenden.
Dieses Feature kann auch beispielsweise mit Face ID bei iOS Geräten verwendet werden.

\paragraph{Umfrage} Die Gruppe hatte zwei Ideen biometrische Identifikation in die Applikation zu integrieren. Ein Vorschlag war nach dem der Benutzer sich eingeloggt und SSO ausgewählt hat, kann der Fingerabdruck zur Anmeldung verwendet werden.
Eine andere Idee war vor jeder Anfrage die gestellt wurde die biometrischen Daten überprüft werden.
\\
Es wurden dafür 10 verschiedene Personen im Alter von 16 bis 20 Jahren befragt. Das Resultat hatte ergeben, dass 8 von 10 Personen bei jeder Ticketanfrage einen Identifikationsprozess durchführen.
Des weiteren wurde gefragt ob beide Vorschläge realisiert werden sollen. Hier haben 10/10 darauf eingestimmt das nur eines dieser Vorhaben realisiert werden soll. Der Grund war zu meist, dass eine zu häufige Abfrage störrend sein kann.

\paragraph{Umsetzung} Die Biometrie wurde jetzt so umgesetzt, dass vor einer Anfragenstellung der Fingerabdruck überprüft wird.

\section{Passwortzurücksetzung}
Auf fast jeder Website, ist es ein Standard eine Passwortzurücksetzung anfordern zu können. In den meisten Fällen wird diese mittels Link unter der Anmeldung gekennzeichnet, beispielsweise mit der Bezeichnung "Passwort vergessen?".
In diesem Projekt haben wir das Versenden einer Zurücksetzungsmail mit SendGrid. Dies ist ein Anbieter dieser Anbieter macht es möglich leicht E-Mails mittels Node.js zu versenden.
Für das EMS Projekt wurde der kostenlose Tarif gewählt. Dieser kann bis zu 100 Mails an einem Tag automatisiert versenden.

\subsection{Anfrage}
Wenn die Person das Passwort zu ihrem/seinem Account vergessen hat, kann diese die E-Mail Adresse eingeben. Wenn auf den Knopf ""

\subsection{Zurücksetzung}

