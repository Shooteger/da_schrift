\chapter{Ticketsysteme}
\reiter

  \section{Ticketsysteme}
  
  %Quellen: https://www.smartsheet.com/how-use-smartsheet-it-ticketing-system https://de.wikipedia.org/wiki/Issue-Tracking-System 
  \subsection{Definition}
    Ein Ticketsystem, auch Issue-Tracking-System oder Service-Tracking-System, hat die Aufgabe, alle eingehenden Anfrage von internen und externen Stellen zu verwalten. IT-Abteilungen haben ein sehr hohes Volumen von Meldungen, die bewältigt werden müssen. Somit wird ein System benötigt, dass jede einzelne Meldung während des gesamten Lebenszyklus von Eingang bis hin zur Lösung verwaltet. \\
Es werden verschiedene Dienste bereitgestellt die von einfacher Intake-Software bis hin zu anspruchsvollen Tools, mit denen Probleme ermittelt und verfolgt werden können, reichen. \\
Ticketsysteme sollen IT-Teams über den Status von Tickets, also gemeldete und formell festgehaltene Probleme, auf dem Laufenden halten und Ihnen helfen, Problemlösungen für diese zu entwickeln. 
Ein Ticket enthält in der Grundform ein technisches Problem, eine Beschreibung des Problems und gegebenenfalls Informationen, die es ermöglichen, dieses Problem zu replizieren. \\
Ticketsysteme ermöglichen es der IT-Abteilung effizienter zu funktionieren, indem alle Informationen über Probleme, für die die IT-Abteilung Verantwortung trägt, so schnell wie möglich zu lösen, in einem zentralen Datenspeicher in ausreichender Form beschrieben gespeichert werden. \\
Erfolgreiches Issue-Tracking kann einem Unternehmen helfen Entwicklungskosten von Software zu verringern, da Probleme die frühzeitig behandelt werden einfacher zu beheben sind als jene, die durch fehlende Infrastruktur eines Tracking-Systems erst später behandelt werden können. \\
  
  \subsection{Funktionsumfang eines Ticketsystems}
  
  \begin{itemize}
	  \item 	\textbf{Self-Service-Portal}: Soll einen One-Stop-Shop darstellen, in dem Kunden und Mitarbeiter ihre Tickets schnell und einfach an die zuständige Abteilung des Unternehmens schicken können. Self-Service-Portale vereinfachen den Kontakt zwischen Kunden bzw. Mitarbeiter und der zuständigen Stelle, indem es genau eine Kontaktstelle gibt. Somit können unnötige Anrufe und E-Mail-Verkehre vermieden werden und optimieren somit den Workflow.  \\
	  \item \textbf{	Organisationssystem:} Nach Eingang eines Tickets in das System ist der erste Schritt, dieses zu protokollieren. In der Regel werden den Tickets eine prägnante Kategorie, eine Dringlichkeit, eine geschätzte Zeit bis zur Lösung, Fristen und eine Beschreibung, um das Problem, dass das Ticket beschreiben soll, reproduzierbar zu machen, zugewiesen. Die Tickets werden dann in ein Organisationssystem gespeichert und für die Abarbeitung zur Verfügung gestellt.  \\
	  \item \textbf{Zuweisen von Tickets:} Um den Workflow des Support-teams zu optimieren, sollten Tickets im Prinzip des One-Stop-Shops, einen „Eigentümer“, also einen Mitarbeiter, der sich ausschließlich um dieses Ticket kümmert, zugewiesen bekommen.  \\
	  \item \textbf{	Sicheres System:} Ticketsysteme sollten sicher sein und somit muss eine hohe Priorität auf die Sicherheit im Auswahlverfahren einer Implementierung gesetzt werden. Informationen, die in einem Ticketsystem gespeichert werden, können interne, systemkritische Probleme beschreiben und müssen damit mit höchster Vertraulichkeit behandelt werden. \\
	  \item \textbf{Live Support:} Ein rund um die Uhr besetzter Helpdesk mit Live-Chat Funktionalität vereinfacht die Meldung von Problemen und fördert somit die Bereitschaft von Kunden und Mitarbeitern aufgetretene Fehler in der Software zu melden und in Kooperation mit dem Support-team in einer ausreichenden Form zu beschreiben. Der Helpdesk kann bei technischen Problemen, die keine Softwareänderung vermögen, oder auf Seiten des Kunden oder Mitarbeiters entstanden sind mit bereits bekannten Workarounds helfen. Ebenfalls können redundante Tickets vermieden werden, indem das Support-team kein neues Ticket schreibt, dessen Problem schon erfasst und formell festgehalten wurde. \\
  \end{itemize}
  
  Die nachstehenden Funktionen können den Support-Betrieb weiter optimieren
  
  \begin{itemize}
			\item \textbf{Unterstützung für mehrere Kanäle:} Es kann ebenfalls sinnvoll sein, andere Kanäle, wie Telefon, Videochat oder E-Mail, zum Informationsaustausch zur Verfügung zu stellen. Komplexe Probleme können leichter gelöst werden, wenn man in direktem Kontakt mit dem Support-Personal steht. Fragen zu zusätzlichen Themen können beantwortet und Probleme leichter geschildert und mit Hilfe des Personals festgehalten werden. Es bleibt dennoch nur ein formaler Prozess zur Erfassung von Problemen und Formalisierung in ein Ticket bestehen. \\
			\item \textbf{Dateianhänge:} Kann zur besseren Verständlichkeit des Tickets beitragen. Besonders bei Problemen technischer Natur kann es Hilfreich sein, wenn Informationen wie Memorydumps oder Log-Dateien für den Support zur Verfügung gestellt werden können. Kann zur besseren Verständlichkeit und der Reproduktion von Fehlern eingesetzt werden. Wenn Dateianhänge zugelassen werden sollen, muss ein Dokumentenverwaltungssystem implementiert werden.\\
			\item \textbf{Mehrsprachige Systeme:} Betrifft hauptsächlich internationale Systeme. Ein Mehrsprachiges Support-System kann dennoch universell sinnvoll sein, da Konversationen in der Muttersprache des Kunden für diesen angenehmer und leichter zu verstehen sind und somit die Bearbeitungszeit der Support-Abteilung verringern können. Internationale Systeme können verschiedensprachige Support-teams einstellen, die die gewonnen Informationen in einer einheitlichen Sprache, meistens Englisch, im Ticketsystem festhalten. Unternehmen, mit kleineren Budgets können Echtzeitübersetzungsdienste implementieren. \\
			\item \textbf{Anpassung:} Es ist Sinnvoll, Ticketsysteme mit Orientierung an Plug-and-Play-Prinzipien zu entwickeln, da somit der Funktionsumfang des Systems an die Anforderungen des Projekts angepasst werden kann und eine spätere Erweiterung des Ticketsystems vereinfacht wird. \\

		\end{itemize}  
  Die genannten Funktionen stellen allein nicht zwingend sicher, dass das Ticketsystem so effektiv wie möglich funktioniert. Es ist wichtig Kunden, Mitarbeitern und Supportern einen Weg einzurichten, Feedback oder Beschwerden zu äußern und auf dieses zu hören und sich aktiv an der Verbesserung und Anpassung des Ticketsystems zu halten. \\
Eine Implementierung eines Berichtssystems, in dem wichtige Metriken und Analysen wie die durchschnittliche Zeit bis zur Problemlösung, Einhaltung von Fristen usw. festgehalten werden kann ebenfalls die Qualität und Effizienz des Supports steigern. \\
Community-Foren und regelmäßige Feedback-Umfragen können beim Aufbau und der Pflege von Beziehungen zwischen Support und Kunden helfen und ein Gefühl von Rechenschaftspflicht und Vertrauen erzeugen. \\

  
  \subsection{Marktanalyse}
  
  
  
  \subsection{Vergleich von Ticketsystemen verschiedener Anbieter}